\section{Projektdurchführung}

	\subsection{Frontend}
	\subsection{Backend}
	\subsubsection{\glslink{OBELISK}{Obelisk}}
	\subsection{DNS}
	Um die Webapplikation bequem von etwaigen Geräten erreichbar zu machen, wurden 4 Subdomains auf den Lufthansa DNS Servern im Intranet angelegt.\\
	Die Interne Domain lautet "dlh.de", somit sind alle Subdomains dieser unterzuordnen. 

	\vspace{8pt}

	\textbf{A-Record (IPv4)}\\
	hcc.dlh.de → xxx.xxx.xxx.xxx (IP Adresse des Servers anonymisiert)[hcc.dlh.de. 86400 IN A xxx.xxx.xxx.xxx]\\

	\textbf{CNAME-Record (Canonical Name)}\\
	foa.dlh.de → hcc.dlh.de [foa.dlh.de. 86400 IN CNAME hcc.dlh.de]\\
	occ.dlh.de → hcc.dlh.de [occ.dlh.de. 86400 IN CNAME hcc.dlh.de]\\
	iocc.dlh.de → hcc.dlh.de [iocc.dlh.de. 86400 IN CNAME hcc.dlh.de]\\

	\subsection{Netzwerkfreigaben}