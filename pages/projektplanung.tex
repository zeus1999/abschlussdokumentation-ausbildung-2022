\section{Projektplanung}

	\subsection{Bewertung des Ist-Zustandes}
	Während der Analyse des Ist-Zustandes wurde festgestellt, dass die bereits bestehenden Systeme für eine aktive Flugsteuerung nicht ausreichend sind. Mit den bestehenden Systemen ist es unmöglich einen Einblick in die Performance des aktuellen Tages zu bekommen. Diese Systeme sind nur auf Vortagswerte ausgelegt. Für eine aktive Flugsteuerung ist es nötig, die Prozesse in Echtzeit abzubilden. Zusätzlich sind beide Systeme, aufgrund ihres hohen Alters, inperformant, man muss mehrere Minuten auf eine simple Auswertung warten. Beide Systeme stellen keine mobile Ansicht zur Verfügung.


	
	\subsection{Nutzwertanalyse}
	Für die Bewertung werden Punkte von \textbf{1-5} vergeben, je höher die Zahl, desto positiver wird das Kriterium von dem Produkt erfüllt.

		\subsubsection{Gewichtungen}

		{
			\underline{Echtzeitfunktion}\\
			Das Thema Echtzeitfunktion ist eines der Hauptgründe für das Projekt. Für eine aktive Flugsteuerung ist es notwendig Livedaten über Flüge und aggregierte Werte zu bekommen, ohne aufwändige manuelle Analysen machen zu müssen. Aus diesem Grund wird die Gewichtung auf \textbf{25 \%} gesetzt.
		
		}

		\vspace{8pt}

		{
			\noindent
			\underline{Mobile Verfügbarkeit}\\
			Die Mobile Verfügbarkeit ist der zweite Hauptgrund warum dieses Projekt ins Leben gerufen wurde. Es ist notwendig, dass auch Mitarbeiter auf dem Vorfeld ohne Laptop von ihrem Handy aus auf das Reporting- und Echtzeittool zugreifen können (KPI in der Hosentasche). Aus diesem Grund wird die Gewichtung ebenfalls auf \textbf{25 \%} gesetzt.
		}

		\vspace{8pt}

		{
			\noindent
			\underline{Performance}\\
			Das Thema Performance ist in der heutigen Zeit ein wichtiges Thema, da die Technik und die Rechenleistung immer besser geworden ist, so möchte man nicht allzu lange auf seine Auswertung warten. Zusätzlich muss die Performance für ein Echtzeitsystem sehr hoch sein, um aktuelle Zahlen zeitgerecht zu liefern. Aus diesem Grund wird die Gewichtung auf \textbf{10 \%} gesetzt.
		}

		\vspace{8pt}

		{
			\noindent
			\underline{Benutzerfreundlichkeit}\\
			Für moderne Webseiten wird die Benutzerfreundlichkeit Groß geschrieben. Eine gut bedienbare und übersichtliche Weboberfläche wirkt sich positiv auf die Nutzer und deren Produktivität aus. Aus diesem Grund wird die Gewichtung auf \textbf{15 \%} gesetzt.
		}

		\vspace{8pt}

		{
			\noindent
			\underline{Anpassbarkeit}\\
			Um ein bereits bestehendes Tool zu erweitern, ist es von Vorteil wenn dieses Modular aufgebaut ist. Für eine aktive Steuerung ist es ebenfalls notwendig das es möglich ist kurzfristige Änderungen schnell umzusetzen um die operativen Mitarbeiter mit den benötigten Reportingtools auszustatten. Aus diesem Grund wird die Gewichtung ebenfalls auf \textbf{15 \%} gesetzt.
		}

		\vspace{8pt}

		{
			\noindent
			\underline{Kosten}\\
			Die Kosten spielen ebenfalls eine entscheidende Rolle, vorallem da das Projekt während der Coronazeit stattfand und dort die Ressourcen noch knapper als unter Normalbedingungen sind. Die Gewichtung wird auf \textbf{10 \%} gesetzt.
		}

		\subsubsection{Auswertung}
		Nach einer ausführlichen Analyse der bestehendes Produkte und des geplanten Projektes wurde folgendes Ergebnis erziehlt.

		\begin{itemize}
			\item Platz 1: Flight Operation Analyser (400 Punkte)
			\item Platz 2: Tableau (165 Punkte)
			\item Platz 3: \glslink{OBELISK}{Obelisk} (135 Punkte)
		\end{itemize}

		\textbf{Für eine ausführliche Nutzwertanalyse: (Appendix: \ref{appendix:nutzwertanalyse})}

	\subsection{Definition von Zielen}
	Nach der Bewertung des Ist-Zustandes wurden folgende Ziele festgelegt.
	\begin{itemize}
		\item Historische Auswertungen
		\item Realtime Funktion
		\item Mobile Ansicht
		\item Erreichbarkeit außerhalb des Lufthansa Netzwerkes
	\end{itemize}


	\subsection{Zeit- und Ressourcenplanung}

	\subsubsection{Zeitplanung}
	Für die Umsetzung des Projektes standen 70 Stunden zur Verfügung, wie es die \glslink{IHK}{IHK} Darmstadt vorschreibt.\footcite{2} Bevor mit dem Projekt gestartet wurde, fand eine Aufteilung auf verschiedene Phasen statt,
	die den kompletten Prozess der Softwareentwicklung abdecken.\\

	\begin{table}[htp]

		\begin{center}
			\begin{tabular}{llll} \toprule
				Phase & Geplant\\ \bottomrule
				Analysephase & 4 h \\
				Entwurfsphase & 16 h \\
				Implementierungsphase & 27 h \\
				Abnahme- und Deploymentphase & 12 h \\
				Dokumentationsphase & 10 h \\ \bottomrule
				Summe & 69 h \\
			\end{tabular}
		\end{center}
		%\caption{Really long caption with lots of info}
	Siehe Appendix: \ref{appendix:zeit_detail} für eine ausführliche Gliederung
	\end{table}


	\subsubsection{Personalplanung}
	Das Projekt wurde mit der Hilfe von Matthias Partzsch geplant und realisiert. Die unten aufgeführten Personenkreise haben regelmäßig das Interface auf ihre Bedienbarkeit und Funktionen getestet.
\\

	\begin{table}[htp]

		\begin{center}
			\begin{tabular}{llll} \toprule
				Beteiligte Person / Personenkreis & Rolle\\ \bottomrule
				Tobias Jung & Projektmitarbeiter \\
				Matthias Partzsch & Projektleiter \\
				Hub Duty Manager & Nutzer (UX Tester) \\
				Hub Duty Officer & Nutzer (UX Tester) \\ \bottomrule
			\end{tabular}
		\end{center}
		%\caption{Really long caption with lots of info}
	\end{table}


	\subsection{Kostenplan}

	\subsubsection{Personalkosten}
	In die Personalkostenplanung fließen nur meine tatsächlichen 69.5 Arbeitsstunden ein, da ich der einzige war, der an dem Projekt aktiv gearbeitet hat.\\

	\begin{table}[htp]

		\begin{center}
			\begin{tabular}{llll} \toprule
				Position & Wert \\ \bottomrule
				Kosten pro Stunde & 9.42 Euro \\
				Umgesetzte Arbeitsstunden & 69.5h \\ \bottomrule
			\end{tabular}
		\end{center}
		%\caption{Really long caption with lots of info}
	\end{table}
	
	\[ 69.5h\,*\,9.42\,\frac{Euro}{h} = 654.69\,Euro \]


	\subsubsection{Sonstige Kosten}
	In die Sonstigen Kosten fließen die Server (Hardware / Betriebssystem) und Lizenzkosten (MongoDB) ein. Gerechnet wird mit einer Nutzungsdauer von 24 Monaten (2 Jahre).\\

	\begin{table}[htp]

		\begin{center}
			\begin{tabular}{llll} \toprule
				Position & Kosten pro Monat \\ \bottomrule
				ThinkStation & 75 Euro \\
				MongoDB Cloud Database & 60 Euro \\ \bottomrule
			\end{tabular}
		\end{center}
		%\caption{Really long caption with lots of info}
	\end{table}
	
	\[ (75 Euro + 60 Euro)\,*\,24 = 3240 Euro \]

	\subsubsection{Gesamtkosten}
	Gerechnet für eine 24 monatige Laufzeit
		
	\[ (Personalkosten + Sonstige Kosten) = Gesamtkosten \]
	\[ (654.69 Euro + 3240 Euro) = 3694.64 Euro \]