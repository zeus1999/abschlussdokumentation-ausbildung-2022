\section{Einleitung}

	\subsection{Projektumfeld}

	Die Lufthansa Systems GmbH \& Co. KG ist mit etwa 2.400 Mitarbeitern primärer IT-Dienstleister und hunderprozentige Tochergesellschaft der Deutschen Lufthansa AG.\footcite{1}
		
	Die Deutsche Lufthansa AG beeinhaltet die Lufthansa Passage Airline, welche das gesamte Fluggeschäft umfasst.

	Die Projektdurchführung erfolgt bei Lufthansa Passage in der Flugsteuerung [FRA L/GS - Operational Steering \& \gls{HCC} (HUB Control Center Frankfurt)].

	Die Verantwortung für den weltweiten Verkehrssteuerungsprozess der Lufthansa wird von der
	Abteilung FRA L/GS wahrgenommen. Dazu gehören insbesondere die Koordination, das
	Monitoring und die Steuerung aller stationsrelevanten Bodenprozesse für Lufthansa und deren
	Handling Partner. Der Fokus liegt hierbei sowohl auf der Pünktlichkeit als auch auf der
	Wirtschaftlichkeit.


	\subsection{Projektbeschreibung}
	Unter der Bezeichnung "Flight Operation Analyser" kurz FOA, soll eine neue Webapplikation geschaffen werden, welche zur aktiven Steuerung beiträgt.

	Die Applikation soll mehrere Funktionen vereinen. Es sollen nicht nur Historische Flugdaten analysiert werden können, sondern die Kennzahlen sollen auch auf tagesaktueller Basis in Form eines "Realtime Tickers" visualisiert werden können.


	\subsection{Projektbegründung}
	Es gibt im Lufthansa Umfeld keine vergleichbare Software, die die benötigten Funktionen vollständig abbilden kann. (Siehe Ist-Analyse).



	\subsection{Projektschnittstellen}
	Um an die benötigten Daten zu gelangen, muss mit mehreren Schnittstellen kommuniziert werden. Dazu zählt eine Oracle-Datenbank auf der die meisten operationellen Daten liegen, aber auch externe \gls{REST} Schnittstellen, wie zu Fraport \gls{BVD} und externen Dienstleistern welche z.B. Aktienkurse abbilden.
