\section{Projektdefinition}
	In der Projektdefinition werden zunächst alle projektrelevanten Rahmenbedingungen erfasst.
	Die Definition beinhaltet eine Analyse des Ist-Zustands und des Weiteren eine Beschreibung
	des Projektumfelds.

	\subsection{Projektumfeld}

		
		
		Die Projektdurchführung erfolgt bei nach Ausleihung durch die Lufthansa Systems GmbH an die Lufthansa Passage in der Flugsteuerung [FRA L/GS - Operational Steering \& \glslink{HCC}{HCC} (HUB Control Center Frankfurt)].

		Die Verantwortung für den weltweiten Verkehrssteuerungsprozess der Lufthansa wird von der
		Abteilung FRA L/GS wahrgenommen. Dazu gehören insbesondere die Koordination, das
		Monitoring und die Steuerung aller stationsrelevanten Bodenprozesse für Lufthansa und deren
		Handling Partner. Der Fokus liegt hierbei sowohl auf der Pünktlichkeit als auch auf der
		Wirtschaftlichkeit.

		\subsubsection{Technisches Umfeld}
		
		Für die Umsetzung der Projektarbeit steht ein Lenovo ThinkPad T490 mit einem Intel i7-8665U (4x 2,11 GHz \glslink{HT}{HT}) und 16 GB \glslink{SODIMM}{SODIMM} \glslink{RAM}{RAM} zur Verfügung.\\
		Für die Entwicklung des Backendes steht NodeJS mit einer Datenbankanbindung an MongoDB zur Verfügung. Als \glslink{IDE}{IDE} wird Visual Studio Code verwendet.
		\\
		
		Als Server steht eine Lenovo ThinkStation zu verfügung.\\
	........

		\subsection{Ist-Analyse}
		Die nachfolgende Ist-Analyse zeigt den groben Aufbau der Reportingtools der Lufthansa.\\

		Es gibt derzeit zwei große Systeme in der Lufthansa, das erste ist Obelisk, welches nicht nur das Datawarehouse von Lufthansa ist, sondern ebenfalls noch eine umfassende Weboberfläche bietet. Obelisk stammt aus dem Jahre 2010.
		Diese Oberfläche hat dutzende Views, auf denen man unterschiedlichste Flugwerte sich anzeigen lassen kann. Die wichtigste View für die Projektabteilung ist, der \glslink{HCC}{HCC}-Reporter. Dieser ist dafür da Flugvolumenzahlen, wie die Passagieranzahl und Flüge anzuzeigen.
		(Appendix: \ref{appendix:obelisk_hcc_reporter})
		Obelisk bietet die möglich mehrere vordefinierte \glslink{SQL}{SQL} Skripte an, welche im Browser ausgeführt werden können. (Appendix: \ref{appendix:obelisk_explorer}).
		Zuästzlich gibt es mehrere Selektionstool, in denen man über eine Grafische Oberfläche \glslink{SQL}{SQL} Statements nachstellen kann. (Appendix: \ref{appendix:obelisk_selektion}).\\

		Das zweite große System ist, Tableau. Tableau ist eine Software für Datenvisualisierung und Reporting. Tableau wird größtenteils für die Prozessanalysierung verwendet. Prozesse sind die einzelnen Abläufe für die Flugzeugabfertigung (Deboarding, Cleaning, Catering, Fueling, Loading, Boarding, Pushback).\\
		
		Übersicht über die DeepDives (Appendix: \ref{appendix:tableau_deepdives})\\
		Beispiel DeepDive - Deboarding (Appendix: \ref{appendix:tableau_deepdive_deboarding})