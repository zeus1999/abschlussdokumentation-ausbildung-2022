\documentclass{article}

\usepackage{graphicx}
\usepackage{svg}
\usepackage{url}
\usepackage{amsfonts}
\usepackage{booktabs}
\usepackage{siunitx}
\usepackage[backend=bibtex, style=verbose, bibstyle=numeric]{biblatex}
\usepackage[toc,title,page]{appendix}
\usepackage{hyperref}
\usepackage{eurosym}
\usepackage[nopostdot]{glossaries}
\usepackage{textcomp}

\hypersetup{
    colorlinks,
    citecolor=,
    filecolor=black,
    linkcolor=black,
    urlcolor=black
}

\addbibresource{literatur.bib}

\renewcommand{\glossarysection}[2][]{}


\makenoidxglossaries

\newglossaryentry{JS}{name=JS, description={Javascript}}
\newglossaryentry{HCC}{name=HCC, description={Hub Control Center}}
\newglossaryentry{REST}{name=REST, description={Representational State Transfer}}
\newglossaryentry{BVD}{name=BVD, description={Bodenverkehrsdienste}}
\newglossaryentry{HT}{name=HT, description={Hyper-Threading}}
\newglossaryentry{SODIMM}{name=SODIMM, description={Small Outline Dual Inline Memory Module}}
\newglossaryentry{RAM}{name=RAM, description={Random-Access Memory}}
\newglossaryentry{IDE}{name=IDE, description={Integrated Development Environment}}
\newglossaryentry{IHK}{name=IHK, description={Industrie- und Handelskammern}}
\newglossaryentry{OBELISK}{name=OBELISK, description={Operationelle Betriebsdatenerfassung Lufthansa - Information, Statistik, Kosten [Lufthansa Datawarehouse]}}

\newglossaryentry{SQL}{name=SQL, description={Structured Query Language}}



\begin{document}

	

	\newcommand\pruefungsbewerberVorname{Tobias}
\newcommand\pruefungsbewerberNachname{Gung}
\newcommand\pruefungsbewerberAdresse{Zeppelinstraße 4}
\newcommand\pruefungsbewerberPLZ{65428}
\newcommand\pruefungsbewerberStadt{Rüsselsheim}

\newcommand\ausbildungsbetrieb{Lufthansa Systems GmbH \& Co. KG}
\newcommand\ausbildungsbetriebAdresse{Am Messeplatz 1}
\newcommand\ausbildungsbetriebPLZ{65479}
\newcommand\ausbildungsbetriebStadt{Raunheim}

\newcommand\abschlusspruefungSaison{Winter}
\newcommand\abschlusspruefungJahr{2021/22}

\newcommand\pruefungsZeitraum{24.09.2021 - 24.11.2021}

\newcommand\ausbildung{Fachinformatiker für Anwendungsentwicklung}

\newcommand\ausbildungsbetreuerName{Martin Trzaska}
\newcommand\ausbildungsbetreuerTelefonnummer{+49 151 58922204}

\newcommand\projektname{Flight Operation Analyser}
\newcommand\projektkurzbeschreibung{Moderne Webapplikation zur Echtzeit Flugsteuerung}
	\pagenumbering{gobble}

\begin{figure}[htbp]
	\centering
	\includesvg[width=0.25\textwidth]{assets/ihk-logo.svg}
\end{figure}

\begin{center}
	\fontsize{14pt}{16pt}\selectfont

	Abschlussprüfung\
	\abschlusspruefungSaison\
	\abschlusspruefungJahr
\end{center}

\vspace{10pt}


\begin{center}
	\fontsize{14pt}{16pt}\selectfont
	\ausbildung
\end{center}

\begin{center}
	\fontsize{14pt}{16pt}\selectfont
	Dokumentation zur betrieblichen Projektarbeit
\end{center}


\vspace{16pt}


\begin{center}
	\fontsize{20pt}{20pt}\selectfont

	\textbf{\projektname}
\end{center}

\begin{center}
	\fontsize{14pt}{14pt}\selectfont

	\projektkurzbeschreibung
\end{center}




\vspace{16pt}


\begin{center}
	\fontsize{12pt}{12pt}\selectfont

	\textbf{Zeitraum: \pruefungsZeitraum}
\end{center}

	

\vspace{16pt}


\begin{center}
	\fontsize{12pt}{12pt}\selectfont

	\textbf{Prüfungsbewerber:}
\end{center}

\begin{center}
	\fontsize{12pt}{16pt}\selectfont
	\pruefungsbewerberVorname\
	\pruefungsbewerberNachname
	\\
	\pruefungsbewerberAdresse
	\\
	\pruefungsbewerberPLZ\
	\pruefungsbewerberStadt
\end{center}

\vspace{32pt}

\begin{figure}[htbp]
	\centering
	\includesvg[width=0.55\textwidth]{assets/lhsystems.svg}
\end{figure}




\begin{center}
	\fontsize{12pt}{12pt}\selectfont

	\textbf{Ausbildungsbetrieb:}
\end{center}

\begin{center}
	\fontsize{12pt}{16pt}\selectfont
	\ausbildungsbetrieb
	\\
	\ausbildungsbetriebAdresse
	\\
	\ausbildungsbetriebPLZ\
	\ausbildungsbetriebStadt

	\vspace{16pt}
	\ausbildungsbetreuerName \\
	\ausbildungsbetreuerTelefonnummer \\
\end{center}


	\pagenumbering{Roman}


	
\addcontentsline{toc}{section}{Inhaltsverzeichnis}

\section*{Inhaltsverzeichnis}
\def\contentsname{\empty}
\tableofcontents
	
	\addcontentsline{toc}{section}{Abkürzungsverzeichnis}
\section*{Abkürzungsverzeichnis}


\printnoidxglossaries


	\pagenumbering{arabic}

	\section{Vorwort}

	Die betriebliche Projektarbeit wird im Rahmen der Abschlussprüfung der Ausbildung zum
	Fachinformatiker Anwendungsentwicklung im Winter 2021 bei der Lufthansa Systems GmbH durchgeführt.
	Das Projekt beinhaltet die Erstellung einer begleitenden
	Dokumentation und einer Präsentation mit anschließendem Fachgespräch. Aus Vereinfachungsgründen wird im weiteren Verlauf dieser Dokumentation auf eine geschlechterspezifische
	Differenzierung verzichtet.


	\subsection{Lufthansa Passage Airline}
	Die Lufthansa Passage Airline fortfolgend als "Lufthansa" bezeichnet, ist Deutschlands größte Fluggesellschaft. Die Kranich-Airline wird gemeinhin als Flagcarrier Deutschlands wahrgenommen. Sie wurde 1953 neu gegründet nachdem sie 1951 aufgelöst wurde. Lufthansa hat zwei Drehkreuze (Hubs) in Deutschland, Frankfurt und München.

	\subsection{Lufthansa Systems GmbH}
	Die Lufthansa Systems GmbH \& Co. KG ist mit etwa 2.400 Mitarbeitern in 17 Ländern primärer IT-Dienstleister und hunderprozentige Tochergesellschaft der Deutschen Lufthansa AG.
	Die Lufthansa Systems bietet 350 Kunden ein großes Portfolio an IT-Produkten und Dienstleistungen. Die Produkte zielen darauf ab die Prozesseffizienz zu steigern und das Reiseerlebniss der Passagiere zu verbessern.
	\footcite{1}


	\section{Projektdefinition}
	In der Projektdefinition werden zunächst alle projektrelevanten Rahmenbedingungen erfasst.
	Die Definition beinhaltet eine Analyse des Ist-Zustands und des Weiteren eine Beschreibung
	des Projektumfelds.

	\subsection{Projektumfeld}

		
		
		Die Projektdurchführung erfolgt bei nach Ausleihung durch die Lufthansa Systems GmbH an die Lufthansa Passage in der Flugsteuerung [FRA L/GS - Operational Steering \& \glslink{HCC}{HCC} (HUB Control Center Frankfurt)].

		{
			\noindent
			Die Verantwortung für den weltweiten Verkehrssteuerungsprozess der Lufthansa wird von der
			Abteilung FRA L/GS wahrgenommen. Dazu gehören insbesondere die Koordination, das
			Monitoring und die Steuerung aller stationsrelevanten Bodenprozesse für Lufthansa und deren
			Handling Partner. Der Fokus liegt hierbei sowohl auf der Pünktlichkeit als auch auf der
			Wirtschaftlichkeit.
		}

		\subsubsection{Technisches Umfeld}
		{
			\noindent
			Für die Umsetzung der Projektarbeit steht ein Lenovo ThinkPad T490 mit einem Intel i7-8665U (4x 2,11 GHz \glslink{HT}{HT}) und 16 GB \glslink{SODIMM}{SODIMM} \glslink{RAM}{RAM} zur Verfügung.\\
			Für die Entwicklung des Backends steht NodeJS mit einer Datenbankanbindung an MongoDB zur Verfügung. Als \glslink{IDE}{IDE} wird Visual Studio Code verwendet.
		}
		{
			\noindent
			Als Server steht eine Lenovo ThinkStation P330 tiny mit einem Intel i7-8700T (4x 2,4GHz \glslink{HT}{HT}) und 16 GB RAM zur Verfügung.\\
		}
		
		\newpage

		\subsection{Ist-Analyse}

		{

			\noindent
			Die nachfolgende Ist-Analyse zeigt den groben Aufbau der Reportingtools der Lufthansa.
		}
		{
			 \noindent
			 Es gibt derzeit zwei große Systeme in der Lufthansa, das erste ist \glslink{OBELISK}{Obelisk}, welches nicht nur das Datawarehouse von Lufthansa ist, sondern ebenfalls noch eine umfassende Weboberfläche bietet. \glslink{OBELISK}{Obelisk} stammt etwa aus dem Jahre 2010.
			Diese Oberfläche hat dutzende Views, auf denen man sich unterschiedlichste Flugwerte anzeigen lassen kann. Die wichtigste View für die Projektabteilung ist, der \glslink{HCC}{HCC}-Reporter. Dieser ist dafür da Flugvolumenzahlen, wie die Passagieranzahl und Flüge, anzuzeigen (Appendix: \ref{appendix:obelisk_hcc_reporter}).
			\glslink{OBELISK}{Obelisk} bietet die Möglichkeit mehrerer vordefinierte \glslink{SQL}{SQL} Skripte an, welche im Browser ausgeführt werden können (Appendix: \ref{appendix:obelisk_explorer}).
			Zusätzlich gibt es mehrere Selektionstools, in denen man über eine grafische Oberfläche \glslink{SQL}{SQL} Statements nachstellen kann (Appendix: \ref{appendix:obelisk_selektion}).
		}

		{
			\noindent
			Das zweite große System ist Tableau. Tableau ist eine Software für Datenvisualisierung und Reporting. Tableau wird größtenteils für die Prozessanalysierung verwendet. Prozesse sind die einzelnen Abläufe für die Flugzeugabfertigung (Deboarding, Cleaning, Catering, Fueling, Loading, Boarding, Pushback).\\

		}
		{
			\noindent
			Übersicht über die DeepDives (Appendix: \ref{appendix:tableau_deepdives})\\
			Beispiel DeepDive - Deboarding (Appendix: \ref{appendix:tableau_deepdive_deboarding})
		}

	\subsection{Amortisationsdauer}

	{
		\noindent
		Eine Angabe der Amortisationsdauer ist nur sehr schwer möglich, da so ein Prozess noch nicht in der Form existiert hat. Je nach Verwendungszweck und Auswertungszeiträumen ergeben sich mehrere Amortisationsdauern.

		\[ Amortisationsdauer = \frac{Anschaffungskosten}{Jaehrlicher Rueckfluss} \]

	}
	
	\subsubsection{Analysetool}

	{
		\noindent
		\underline{\textbf{Annahmen}}\\
		Ø Verdienst: 40 Euro/h\\
		Ø Analysezeitraum 1: Wöchentlich von der Vorwoche (Dauer: 1 h)\\
		Ø Analysezeitraum 2: Wöchentlich unterschiedliche Detailanalyse eines 1 Monatigen Zeitraums (3 h)\\
		Ø Analysezeitraum 3: 5x wöchentlich unterschiedliche Detailanalyse eines einwöchigen Zeitraums (jeweils 1 h)\\

	}

	{
		\noindent
		Gesamter Zeitlicher Aufwand pro Woche: 9 h\\
		Gesamter monetärer Aufwand pro Woche: 360 Euro (18.720 Euro / Jahr)

		\[ 9\,h\,*\,40\,\frac{Euro}{h} = 360\,Euro \]
		\[ Amortisationdauer = \frac{3694,64\,Euro}{18.720\,Euro} = 0,19736\,Jahre \]

		Das Projekt amortisiert sich nach 3 Monaten (gerundet).

	}

	
	\subsubsection{Echtzeitsteuerungstool}

	{
		Durch die Echtzeitsteuerung lässt sich die Performance verbessern, wodurch die Wahrscheinlichkeit das ein Flugzeug stehenbleibt oder das ein Passagier einen Anschluss verpasst, sinkt. Da es derzeit keine Daten zur Auswertung gibt, wird im folgendem Schritt mit Beispieldaten gerechnet.\\
		\noindent
		\underline{\textbf{Annahmen}}\\
		Ø Kosten für stehengebliebenen Passagier: 200 Euro\\
		Ø Kosten für stehengebliebenes Flugzeug: 20.000 Euro/h\\
		Gerettete Passagiere pro Jahr: 100\\
		Gerettete Flugzeuge pro Jahr: 2\\

	}

	{
		\noindent
		Vermiedene Kosten durch gerettete Passagiere: 20.000 Euro\\
		Vermiedene Kosten durch gerettete Flugzeuge: 40.000 Euro\\
		Vermiedene Kosten: 60.000 Euro\\

		\[ Amortisationdauer = \frac{3694,64\,Euro}{60.000\,Euro} = 0,0615773\,Jahre \]

		Das Projekt amortisiert sich nach einem Monat (gerundet).

	}
	\section{Projektplanung}

	\subsection{Bewertung des Ist-Zustandes}
	Während der Analyse des Ist-Zustandes wurde festgestellt, dass die bereits bestehenden Systeme für eine aktive Flugsteuerung nicht ausreichend sind. Mit den bestehenden Systemen ist es unmöglich einen Einblick in die Performance des aktuellen Tages zu bekommen. Diese Systeme sind nur auf Vortagswerte ausgelegt. Für eine aktive Flugsteuerung ist es nötig, die Prozesse in Echtzeit abzubilden. Zusätzlich sind beide Systeme, aufgrund ihres hohen Alters, inperformant, man muss mehrere Minuten auf eine simple Auswertung warten. Beide Systeme stellen keine mobile Ansicht zur Verfügung.


	
	\subsection{Nutzwertanalyse}
	Für die Bewertung werden Punkte von \textbf{1-5} vergeben, je höher die Zahl, desto positiver wird das Kriterium von dem Produkt erfüllt.

		\subsubsection{Gewichtungen}

		{
			\underline{Echtzeitfunktion}\\
			Das Thema Echtzeitfunktion ist eines der Hauptgründe für das Projekt. Für eine aktive Flugsteuerung ist es notwendig Livedaten über Flüge und aggregierte Werte zu bekommen, ohne aufwändige manuelle Analysen machen zu müssen. Aus diesem Grund wird die Gewichtung auf \textbf{25 \%} gesetzt.
		
		}

		\vspace{8pt}

		{
			\noindent
			\underline{Mobile Verfügbarkeit}\\
			Die Mobile Verfügbarkeit ist der zweite Hauptgrund warum dieses Projekt ins Leben gerufen wurde. Es ist notwendig, dass auch Mitarbeiter auf dem Vorfeld ohne Laptop von ihrem Handy aus auf das Reporting- und Echtzeittool zugreifen können (KPI in der Hosentasche). Aus diesem Grund wird die Gewichtung ebenfalls auf \textbf{25 \%} gesetzt.
		}

		\vspace{8pt}

		{
			\noindent
			\underline{Performance}\\
			Das Thema Performance ist in der heutigen Zeit ein wichtiges Thema, da die Technik und die Rechenleistung immer besser geworden ist, so möchte man nicht allzu lange auf seine Auswertung warten. Zusätzlich muss die Performance für ein Echtzeitsystem sehr hoch sein, um aktuelle Zahlen zeitgerecht zu liefern. Aus diesem Grund wird die Gewichtung auf \textbf{10 \%} gesetzt.
		}

		\vspace{8pt}

		{
			\noindent
			\underline{Benutzerfreundlichkeit}\\
			Für moderne Webseiten wird die Benutzerfreundlichkeit Groß geschrieben. Eine gut bedienbare und übersichtliche Weboberfläche wirkt sich positiv auf die Nutzer und deren Produktivität aus. Aus diesem Grund wird die Gewichtung auf \textbf{15 \%} gesetzt.
		}

		\vspace{8pt}

		{
			\noindent
			\underline{Anpassbarkeit}\\
			Um ein bereits bestehendes Tool zu erweitern, ist es von Vorteil wenn dieses Modular aufgebaut ist. Für eine aktive Steuerung ist es ebenfalls notwendig das es möglich ist kurzfristige Änderungen schnell umzusetzen um die operativen Mitarbeiter mit den benötigten Reportingtools auszustatten. Aus diesem Grund wird die Gewichtung ebenfalls auf \textbf{15 \%} gesetzt.
		}

		\vspace{8pt}

		{
			\noindent
			\underline{Kosten}\\
			Die Kosten spielen ebenfalls eine entscheidende Rolle, vorallem da das Projekt während der Coronazeit stattfand und dort die Ressourcen noch knapper als unter Normalbedingungen sind. Die Gewichtung wird auf \textbf{10 \%} gesetzt.
		}

		\subsubsection{Auswertung}
		Nach einer ausführlichen Analyse der bestehendes Produkte und des geplanten Projektes wurde folgendes Ergebnis erziehlt.

		\begin{itemize}
			\item Platz 1: Flight Operation Analyser (400 Punkte)
			\item Platz 2: Tableau (165 Punkte)
			\item Platz 3: \glslink{OBELISK}{Obelisk} (135 Punkte)
		\end{itemize}

		\textbf{Für eine ausführliche Nutzwertanalyse: (Appendix: \ref{appendix:nutzwertanalyse})}

	\subsection{Definition von Zielen}
	Nach der Bewertung des Ist-Zustandes wurden folgende Ziele festgelegt.
	\begin{itemize}
		\item Historische Auswertungen
		\item Realtime Funktion
		\item Mobile Ansicht
		\item Erreichbarkeit außerhalb des Lufthansa Netzwerkes
	\end{itemize}


	\subsection{Zeit- und Ressourcenplanung}

	\subsubsection{Zeitplanung}
	Für die Umsetzung des Projektes standen 70 Stunden zur Verfügung, wie es die \glslink{IHK}{IHK} Darmstadt vorschreibt.\footcite{2} Bevor mit dem Projekt gestartet wurde, fand eine Aufteilung auf verschiedene Phasen statt,
	die den kompletten Prozess der Softwareentwicklung abdecken.\\

	\begin{table}[htp]

		\begin{center}
			\begin{tabular}{llll} \toprule
				Phase & Geplant\\ \bottomrule
				Analysephase & 4 h \\
				Entwurfsphase & 16 h \\
				Implementierungsphase & 27 h \\
				Abnahme- und Deploymentphase & 12 h \\
				Dokumentationsphase & 10 h \\ \bottomrule
				Summe & 69 h \\
			\end{tabular}
		\end{center}
		%\caption{Really long caption with lots of info}
	Siehe Appendix: \ref{appendix:zeit_detail} für eine ausführliche Gliederung
	\end{table}


	\subsubsection{Personalplanung}
	Das Projekt wurde mit der Hilfe von Matthias Partzsch geplant und realisiert. Die unten aufgeführten Personenkreise haben regelmäßig das Interface auf ihre Bedienbarkeit und Funktionen getestet.
\\

	\begin{table}[htp]

		\begin{center}
			\begin{tabular}{llll} \toprule
				Beteiligte Person / Personenkreis & Rolle\\ \bottomrule
				Tobias Jung & Projektmitarbeiter \\
				Matthias Partzsch & Projektleiter \\
				Hub Duty Manager & Nutzer (UX Tester) \\
				Hub Duty Officer & Nutzer (UX Tester) \\ \bottomrule
			\end{tabular}
		\end{center}
		%\caption{Really long caption with lots of info}
	\end{table}


	\subsection{Kostenplan}

	\subsubsection{Personalkosten}
	In die Personalkostenplanung fließen nur meine tatsächlichen 69.5 Arbeitsstunden ein, da ich der einzige war, der an dem Projekt aktiv gearbeitet hat.\\

	\begin{table}[htp]

		\begin{center}
			\begin{tabular}{llll} \toprule
				Position & Wert \\ \bottomrule
				Kosten pro Stunde & 9.42 Euro \\
				Umgesetzte Arbeitsstunden & 69.5h \\ \bottomrule
			\end{tabular}
		\end{center}
		%\caption{Really long caption with lots of info}
	\end{table}
	
	\[ 69.5h\,*\,9.42\,\frac{Euro}{h} = 654.69\,Euro \]


	\subsubsection{Sonstige Kosten}
	In die Sonstigen Kosten fließen die Server (Hardware / Betriebssystem) und Lizenzkosten (MongoDB) ein. Gerechnet wird mit einer Nutzungsdauer von 24 Monaten (2 Jahre).\\

	\begin{table}[htp]

		\begin{center}
			\begin{tabular}{llll} \toprule
				Position & Kosten pro Monat \\ \bottomrule
				ThinkStation & 75 Euro \\
				MongoDB Cloud Database & 60 Euro \\ \bottomrule
			\end{tabular}
		\end{center}
		%\caption{Really long caption with lots of info}
	\end{table}
	
	\[ (75 Euro + 60 Euro)\,*\,24 = 3240 Euro \]

	\subsubsection{Gesamtkosten}
	Gerechnet für eine 24 monatige Laufzeit
		
	\[ (Personalkosten + Sonstige Kosten) = Gesamtkosten \]
	\[ (654.69 Euro + 3240 Euro) = 3694.64 Euro \]
	\section{Projektdurchführung}

	\subsection{Frontend}
	Für die Entwicklung des Frontends entschied ich mich für Angular in der Version 12 mit Angular Material als Designbibliothek. Angular ist ein Frontend Webapplikationsframework, welches auf TypeScript basiert. Es wird ständig durch eine Community bestehend aus Tausenden von Einzelpersonen, Unternehmen und Organisationen, angeführt von Google, weiterentwickelt. \footcite{3}\\
	Angular ist komponentenbasiert, somit setzt sich die Applikation aus einer Hauptkompontente zusammen, welche sich wiederum dann aus mehreren Unterkomponenten zusammensetzt. Derzeit sind es 31 Module, welche die Prozesse, den Liveticker und verschiedene Steuermodule enthalten. Das gesamte Projekt ist modular aufgebaut, sodass man alle Module/Bausteine wiederverwenden kann, um somit die Effizienz, Weiterentwickelbarkeit und Wiedererkennung zu verbessern.\\
	Die Filter Komponente, welche eine zentrale Funktion inne hat, ist dafür da, den Datenfluss zwischen Frontend und Backend, aufgrund der Nutzereinstellungen zu steuern (Appendix: \ref{appendix:foa_filter}). Jeder einzelne Filter bietet die Möglichkeit exakte Selektionen für die Auswahl der Flüge zu treffen (Appendix: \ref{appendix:foa_filter_2}). Zusätzlich zu der Selektion, gibt es "vordefinierte Sets (Predefined Sets), womit es möglich ist, Filteroptionen zu gruppieren. Zum Beispiel bei dem Filter "Subfleet", bei dem man den Flugzeugtyp auswählen kann, gibt es die Sets: Narrow Body (schmale Flugzeuge) und Wide Body (breite Flugzeuge), das sorgt für eine bessere User Experience. \ref{appendix:foa_filter_3})

	\subsubsection{Komponentenaufbau}
	{
		\noindent
		Jede Komponente ist gleich aufgebaut, im Projektverzeichnis gibt es einen Unterordner mit dem Namen, in diesem Ordner sind drei Dateien.
	}
	\vspace{10pt}

	{
		\noindent
		Die \textbf{NAME.compontent.ts} ist die zentrale Datei der Komponente welche die Logik beeinhaltet (Appendix: \ref{appendix:component_ts}).
	}
	\vspace{10pt}

	{
		\noindent
		Die \textbf{NAME.compontent.html} beeinhaltet das \glslink{HTML}{HTML} Template welches dann in die Webapplikation eingebunden wird (Appendix: \ref{appendix:component_html}).
	}
	\vspace{10pt}

	{
		\noindent
		Die \textbf{NAME.compontent.scss} beeinhaltet das \glslink{SCSS}{SCSS} Stylesheet welches exclusiv für die HTML Datei der Komponente zuständig ist und nicht das restliche \glslink{DOM}{DOM} beeinflusst (Appendix: \ref{appendix:component_scss}).
	}
	

	

	\subsection{Backend}
	Das Backend wurde in NodeJS umgesetzt. NodeJS ist eine Open-Source-JavaScript-Laufzeitumgebung, welche auf der Google V8 Engine basiert und für hochskalierbare Netzwerkapplikationen entworfen wurde. NodeJS-Anwendungen sind ereignisbasiert und laufen asynchron ab, diese sind lediglich Single Thread fähig, man kann eine Applikation aber "forken", um diese zu replizieren, um sie in mehreren Threads laufen zu lassen.\footcite{4}\\
	Das Backend hat mehrere Funktionen, einmal um eine bestehende OracleDB Datenbankverbindung zum Lufthansa Datawarehouse (\glslink{OBELISK}{Obelisk}) aufrecht zuhalten und verschiedene Tabellen für Flugereignisse und Prozesszeitstempel in der eigenen lokalen MongoDB Datenbank zu replizieren und nachzuberechnen. Dies geschieht alle 30 Sekunden, um eine "nahezu Echtzeit" Auswertung möglich zu machen (Appendix: \ref{appendix:mongoose_model}).



	\subsubsection{\glslink{OBELISK}{Obelisk}}

	{
		\noindent
		\glslink{OBELISK}{Obelisk} ist das Data Warehouse der Lufthansa mit Schwerpunkt auf Daten aus dem operativen Umfeld. \glslink{OBELISK}{Obelisk} wird teilweise auch konzernübergreifend genutzt.
	}

	{
		\noindent
		Das Archiv des Data Warehouses reicht zum Teil über mehrere Jahrzehnte zurück, dessen Verarbeitung aber im Wesentlichen echtzeitnah erfolgt.
	}

	{
		\noindent
		\glslink{OBELISK}{Obelisk} basiert auf einem Oracle Enterprise RDBMS, neben dem ein Apache Webserver aufgesetzt wurde, um eine webbasierte Plattform anzubieten. Abfragen über Business Intelligence Tools wie IBM Cognos sind ebenfalls im Programm enthalten.
	}

	{
		\noindent
		\textbf{Aus Sicherheitsgründen werden die interne Struktur und die verwendeten Tabellen nicht offengelegt.}
	}

	\subsection{DNS}

	{
		\noindent
		Um die Webapplikation bequem von etwaigen Geräten erreichbar zu machen, wurden 4 Subdomains auf den Lufthansa DNS Servern im Intranet angelegt.\\
		Die Interne Domain lautet "dlh.de", somit sind alle Subdomains dieser unterzuordnen.	
	}

	{
		\noindent
		IP Adressen werden aus Sicherheitsgründen anonymisiert.
	}
	
	\vspace{16pt}

	{
		\noindent
		\textbf{A-Record (IPv4)}\\
		hcc.dlh.de → xxx.xxx.xxx.xxx [hcc.dlh.de. 86400 IN A xxx.xxx.xxx.xxx]
	}

	\vspace{8pt}

	{
		\noindent
		\textbf{CNAME-Record (Canonical Name)}\\
		foa.dlh.de → hcc.dlh.de [foa.dlh.de. 86400 IN CNAME hcc.dlh.de]\\
		occ.dlh.de → hcc.dlh.de [occ.dlh.de. 86400 IN CNAME hcc.dlh.de]\\
		iocc.dlh.de → hcc.dlh.de [iocc.dlh.de. 86400 IN CNAME hcc.dlh.de]
	}
	

	

	\subsection{Netzwerkfreigaben}
	Damit man aus dem Netzwerk auf den Webserver zugreifen kann, ist es zwingend erforderlich in der Lufthansa internen Firewall die Ports für bestimmte Nutzergruppen zu öffnen.
	Die Anforderung war, dass jeder aus dem Lufthansa Netz und über VPN auf den Webserver zugreifen kann, glücklicherweise geht jeder Traffic über den Lufthansa Proxy, weshalb lediglich die Portfreigabe zwischen dem Proxy Server und dem Webserver bestehen muss.
	
	Folgende Ports werden aufgrund des Webservers geöffnet:

	\begin{itemize}
		\item 80/TCP - HTTP Server (Leitet auf HTTPS weiter)
		\item 443/TCP - HTTPS Server
	\end{itemize}


	\nocite{*}
	\addcontentsline{toc}{section}{Quellenverzeichnis}
	\printbibliography[type=online,resetnumbers=true]

	\pagenumbering{roman}

	\appendix
\section{Appendix: Detaillierte Zeitplanung - Soll zu Ist}

\begin{tabular}{llll} \toprule
	Phase & Geplant & Tatsachlich & Differenz\\ \bottomrule
	Analyse der bestehenden Systeme & 3 h & 2 h & -1 h \\
	Bewertung des Ist-Zustandes & 1 h & 0.5 h & -0.5 h \\
	Definition von Zielen & 3 h & 2 h & -1 h \\
	Zeit- und Ressourcenplanung & 2 h & 2 h & 0 h \\
	Auseinandersetzung mit Ops Kollegen der Flugsteuerung & 3 h & 3 h & 0 h \\
	Aufstellen von Style Guidelines & 1 h & 1 h & 0 h \\
	Skizzierung eines ersten Entwurfes & 1 h & 1 h & 0 h \\
	Planung der Backendstruktur (mit Technologien) & 3 h & 2 h & -1 h \\
	Planung der Frontendstruktur (mit Technologien) & 3 h & 2 h & -1 h \\
	Programmierung eines Prototypen (Frontend) & 10 h & 15 h & 5 h \\
	Programmierung eines Prototypen (Backend) & 10 h & 15 h & 5 h \\
	Vorstellung des Prototypen & 2 h & 1 h & -1 h \\
	Anpassung der Bedienung und Fehlerbehebung & 5 h & 3 h & -2 h \\
	Anbindung an die Datenbank & 2 h & 1 h & -1 h \\
	Testen der Datenbankverbindung & 2 h & 1 h & -1 h \\
	Code Sichtung und Cleanup + Fehlerbehebung & 5 h & 5 h & 0 h \\
	Live Schaltung der Applikation und Monitoring & 3 h & 1 h & -2 h \\
	Dokumentation des Projektes & 10 h & 12 h & 2 h \\ \bottomrule

	Summe & 69 h & 69.5 h & 0.5 h \\
\end{tabular}


\section{Title of Appendix B}
% the \\ insures the section title is centered below the phrase: Appendix B

Text of Appendix B is Here

	
\end{document}
